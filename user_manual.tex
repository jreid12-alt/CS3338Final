\documentclass[12pt]{article}
\usepackage{graphicx}
\usepackage{hyperref}
\usepackage{amsmath}

\title{User Manual for Metro Bike Share}
\author{}
\date{}

\begin{document}

\maketitle

\section*{Table of Contents}
\begin{enumerate}
    \item Overview of Metro Bike Share
    \item Getting Started
    \begin{enumerate}
        \item Accessing the Web Application
        \item Installing and Using the Android App
    \end{enumerate}
    \item Core Features
    \begin{enumerate}
        \item Real-Time Bike Availability
        \item Route Planning \& Directions
        \item Historical Data Visualization
        \item Accident Visualization
    \end{enumerate}
    \item User Interface Walkthrough
    \begin{enumerate}
        \item Web App Interface
        \item Android App Interface
    \end{enumerate}
    \item Managing Your Account
    \begin{enumerate}
        \item Creating and Managing Your Account
        \item Viewing Trip History (Web and Android)
    \end{enumerate}
    \item Advanced Features
    \begin{enumerate}
        \item Customizing Bike Routes
        \item Filtering Accident Locations
        \item Exporting Data
    \end{enumerate}
    \item Troubleshooting
    \item FAQs
    \item Contact Support
\end{enumerate}

\section{Overview of Metro Bike Share}
Metro Bike Share is a bike rental system providing convenient, eco-friendly transportation options in urban areas. The system offers both \textbf{real-time bike availability} and \textbf{historical data visualization} through the \textbf{Web Application} and \textbf{Android App}. Additionally, users can access \textbf{route planning} and \textbf{accident data} to improve cycling safety.

\subsection*{Key Features}
\begin{itemize}
    \item \textbf{Real-Time Bike Availability}: See available bikes at nearby stations.
    \item \textbf{Route Planning}: Plan your route and get bike directions.
    \item \textbf{Accident Visualization}: View historical bike accident data on maps.
    \item \textbf{Historical Data}: Access CSV data for bike usage and trends.
\end{itemize}

\section{Getting Started}
\subsection{Accessing the Web Application}
\begin{enumerate}
    \item Open a web browser and go to the \textbf{Metro Bike Share Website}.
    \item If you don’t have an account, click on \textbf{Sign Up} and follow the instructions to create an account.
    \item Log in using your credentials (username/email and password).
    \item After logging in, you can begin accessing real-time bike availability, route planning, and historical data visualizations.
\end{enumerate}

\subsection{Installing and Using the Android App}
\begin{enumerate}
    \item \textbf{Download the App}: Visit the \textbf{Google Play Store} and search for “Metro Bike Share.” Click \textbf{Install} to download and install the app on your Android device.
    \item \textbf{Create an Account}: Open the app and select \textbf{Sign Up} to create a new user account. Alternatively, if you already have an account, click \textbf{Login}.
    \item \textbf{Getting Started with the App}: After logging in, you can begin viewing available bikes, planning routes, and accessing accident data directly on the map.
\end{enumerate}

\section{Core Features}

\subsection{Real-Time Bike Availability}
\begin{itemize}
    \item \textbf{Web App}: On the main dashboard, view the interactive map with bike stations marked as pins. The stations will show the number of available bikes in real-time.
    \item \textbf{Android App}: The home screen displays a map with available bikes at nearby stations. Tap on any station to view detailed information, including bike availability, station capacity, and station location.
\end{itemize}

\subsection{Route Planning \& Directions}
\subsubsection*{Web App}
From the main interface, input your \textbf{Starting Location} and \textbf{Destination} in the provided fields. The system will provide \textbf{bike-friendly routes} using the \textbf{Google Maps Directions API} or the \textbf{ArcGIS API} for optimal bike paths. You can view the \textbf{distance}, \textbf{duration}, and \textbf{route details}, including step-by-step instructions.

\subsubsection*{Android App}
On the \textbf{Route Planning Screen}, enter your starting point and destination. The app will display available routes, including information on \textbf{estimated travel time} and \textbf{route elevation}. The app also decodes \textbf{polyline paths} to show the route on the map.

\subsection{Historical Data Visualization}
\subsubsection*{Web App}
You can access \textbf{historical bike usage data} by selecting the "Historical Data" section. This will allow you to visualize trends such as:
\begin{itemize}
    \item \textbf{Total number of bike rides per station}
    \item \textbf{Popular routes}
    \item \textbf{Bike availability over time}
    \item Use interactive graphs to filter by date range and visualize data in \textbf{CSV} format.
\end{itemize}

\subsubsection*{Android App}
Access historical data visualization via the \textbf{Data tab}. The app will show trends like the \textbf{number of rides}, and you can filter by specific stations or time frames.

\subsection{Accident Visualization}
\subsubsection*{Web App}
Access \textbf{Bicycle Accident Data} from the "Accident Visualization" section on the map. The system displays:
\begin{itemize}
    \item \textbf{Accident hotspots} (areas with frequent accidents).
    \item \textbf{Incident markers} showing where accidents occurred, along with details such as time, location, and severity.
\end{itemize}

\subsubsection*{Android App}
The \textbf{Accident Map} shows real-time and historical accident data as markers or heatmaps. Users can tap on accident locations to get more details about the incident.

\section{User Interface Walkthrough}

\subsection{Web App Interface}
\begin{itemize}
    \item \textbf{Map View}: The primary interface shows an interactive map with bike station markers and real-time bike availability.
    \item \textbf{Route Planner}: Users can enter starting and ending locations to plan bike routes.
    \item \textbf{Sidebar}: Displays bike availability statistics, accident data, and historical data options.
    \item \textbf{Navigation Bar}: Allows access to various features like \textbf{Account Settings}, \textbf{History}, and \textbf{Support}.
\end{itemize}

\subsection{Android App Interface}
\begin{itemize}
    \item \textbf{Home Screen}: Displays a map showing bike stations with available bikes.
    \item \textbf{Route Planning}: Tap on the "Plan Route" button to input your journey and see bike-friendly routes.
    \item \textbf{Data Tab}: Access historical data visualizations for bike trends and accident hotspots.
    \item \textbf{Menu}: Provides access to \textbf{Account Settings}, \textbf{Help \& Support}, and \textbf{Logout}.
\end{itemize}

\section{Managing Your Account}

\subsection{Creating and Managing Your Account}
\begin{enumerate}
    \item \textbf{Sign Up}: On both the web and Android app, click on the \textbf{Sign Up} button, provide your personal information, and create a password.
    \item \textbf{Login}: Use your email and password to sign in.
    \item \textbf{Profile Settings}: You can update your profile, change your password, and view your \textbf{trip history} and \textbf{bike usage statistics} under \textbf{Account Settings}.
\end{enumerate}

\subsection{Viewing Trip History (Web and Android)}
\begin{itemize}
    \item \textbf{Web App}: In the "History" section, you can view all trips taken, including details like start and end times, stations used, and trip duration.
    \item \textbf{Android App}: The trip history is available in the \textbf{History Tab}, where you can filter trips by date and station.
\end{itemize}

\section{Advanced Features}

\subsection{Customizing Bike Routes}
\begin{itemize}
    \item \textbf{Web App}: Use the \textbf{Route Planner} to adjust your journey. You can select preferences such as:
    \begin{itemize}
        \item Avoiding certain areas
        \item Customizing stops along the way
        \item Selecting faster or more scenic routes
    \end{itemize}
    \item \textbf{Android App}: After entering your destination, customize your route by selecting from suggested paths or adjusting based on bike lane availability.
\end{itemize}
\end{document}
