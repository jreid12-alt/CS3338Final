\documentclass[12pt]{article}
\usepackage{graphicx}
\usepackage{hyperref}
\usepackage{amsmath}
\usepackage{enumerate}

\title{Design Specification for Metro Bike Share Project}
\author{}
\date{}

\begin{document}

\maketitle

\section*{Project Breakdown}
The following section provides a detailed breakdown of every page, part, and tool involved in the Metro Bike Share project.

\subsection*{1. Web Application}
\begin{itemize}
    \item \textbf{Homepage}: Displays the map of available bike stations in real-time, showing bike availability and station locations. Allows users to plan routes, view accident data, and check historical data.
    \item \textbf{Route Planning Page}: A page where users can input their starting and ending locations and get bike-friendly routes. Includes distance, duration, and step-by-step instructions for the journey.
    \item \textbf{Historical Data Visualization Page}: Displays charts and graphs for historical bike data such as usage trends, popular routes, and bike availability over time. Allows users to filter by date ranges.
    \item \textbf{Accident Visualization Page}: Displays accident hotspots and markers on a map, allowing users to view accident data on bike routes.
    \item \textbf{Account Settings Page}: Allows users to view and edit their account information, including trip history and personal settings.
\end{itemize}

\subsection*{2. Android Application}
\begin{itemize}
    \item \textbf{Home Screen}: Shows the map with available bikes and stations, allowing users to view real-time bike availability.
    \item \textbf{Route Planning Screen}: Allows users to input starting and ending points to receive bike-friendly routes. Includes additional features such as adjusting routes based on bike lane availability.
    \item \textbf{Historical Data Screen}: Displays data visualizations for bike trends and accident hotspots. Users can filter data by time or location.
    \item \textbf{Menu Screen}: Provides access to account settings, help/support, and logout options.
\end{itemize}

\subsection*{3. Tools and APIs}
\begin{itemize}
    \item \textbf{Google Maps JavaScript API}: Used for rendering maps, calculating routes, and displaying bike stations.
    \item \textbf{ArcGIS JavaScript API}: Used for displaying and analyzing spatial data, including accident hotspots and historical bike data.
    \item \textbf{CSV Data Handling}: Used for importing, manipulating, and visualizing Metro Bike Share historical data.
    \item \textbf{Firebase/Backend Services}: Used for managing user data, including login, trip history, and real-time bike availability.
\end{itemize}

\newpage

\section*{Snapshot Objectives}

\subsection*{i. Start Objective}
At the start of the project, the primary objectives were to establish a clear understanding of the Metro Bike Share system and its real-time functionality. This included:
\begin{itemize}
    \item Familiarizing with the Google Maps and ArcGIS APIs to implement maps and routes.
    \item Understanding how to fetch and display real-time data about bike availability at stations.
    \item Designing wireframes for the web and mobile applications.
    \item Setting up the basic backend architecture for handling user data, including authentication and bike station data.
\end{itemize}
These objectives were crucial for laying the foundation for the project and ensuring that future tasks would align with the goals of real-time bike sharing and visualization.

\subsection*{ii. 1st Checkpoint Objective}
By the first checkpoint, the following tasks were completed:
\begin{itemize}
    \item Implemented the core structure for the web application and Android app.
    \item Integrated the Google Maps API to display real-time bike availability on the map.
    \item Established backend services for user authentication and session management.
\end{itemize}
New tasks added at this stage include:
\begin{itemize}
    \item Implementing route planning features in both the web and Android apps, allowing users to input start and end locations.
    \item Initiating the historical data page for the web app, where users can visualize trends in bike usage.
    \item Integrating accident data into the map, displaying accident hotspots based on historical information.
\end{itemize}

\subsection*{iii. 2nd Checkpoint Objective}
By the second checkpoint, the following tasks were completed:
\begin{itemize}
    \item Finalized the route planning functionality on both the web and Android apps, allowing users to view optimized bike routes.
    \item Completed the historical data visualization page for the web app with CSV file parsing and filtering functionality.
    \item Accident visualization functionality was fully integrated on the web app, with users able to view heatmaps of bike accidents.
\end{itemize}
New tasks added at this stage include:
\begin{itemize}
    \item Implementing a user-friendly interface to manage personal accounts, including trip history and settings.
    \item Enhancing the accident visualization tool with more granular filters (e.g., accident severity, date range).
    \item Finalizing backend systems to handle real-time data updates for bike availability and accident reporting.
\end{itemize}

\subsection*{iv. Due Date Checkpoint Objective}
By the due date, the following tasks were completed:
\begin{itemize}
    \item Both the web and mobile applications were fully functional, with real-time bike station availability, route planning, historical data visualization, and accident visualization working as intended.
    \item Backend services for real-time data synchronization were stable, with minimal latency.
    \item User interface was fully responsive and tested across different devices, ensuring smooth interaction on both desktop and mobile platforms.
\end{itemize}
Future work includes:
\begin{itemize}
    \item Finalizing the integration of advanced features such as predictive bike availability and personalized recommendations.
    \item Conducting a final round of testing and bug fixes across both platforms to ensure stability.
    \item Potentially adding new features based on user feedback, such as in-app bike rental options and expanded geographic coverage.
\end{itemize}

\end{document}

